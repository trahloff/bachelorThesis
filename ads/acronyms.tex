%!TEX root = ../main.tex

\addchap{\langabkverz}
%nur verwendete Akronyme werden letztlich im Abkürzungsverzeichnis des Dokuments angezeigt
%Verwendung: 
%		\ac{Abk.}   --> fügt die Abkürzung ein, beim ersten Aufruf wird zusätzlich automatisch die ausgeschriebene Version davor eingefügt bzw. in einer Fußnote (hierfür muss in header.tex \usepackage[printonlyused,footnote]{acronym} stehen) dargestellt
%		\acs{Abk.}   -->  fügt die Abkürzung ein
%		\acf{Abk.}   --> fügt die Abkürzung UND die Erklärung ein
%		\acl{Abk.}   --> fügt nur die Erklärung ein
%		\acp{Abk.}  --> gibt Plural aus (angefügtes 's'); das zusätzliche 'p' funktioniert auch bei obigen Befehlen
%	siehe auch: http://golatex.de/wiki/%5Cacronym
%	
\begin{acronym}[YTMMM]
\setlength{\itemsep}{-\parsep}

\acro{AGPL}{Affero GNU General Public License}
\acro{WSN}{Wireless Sensor Network}
\acro{MANET}{Mobile wireless Ad-hoc NETwork}
\acro{MAC}{Multiple Access Control}
\acro{QoS}{Quality of Service}
\acro{DSR}{Dynamic Source Routing}
\acro{API}{Application Programming Interface}
\acro{WYSIWYG}{What You See Is What You Get}
\acro{HTML}{HyperText Markup Language}
\acro{PaaS}{Platform-as-a-Service}
\acro{ESP}{Event Stream Processing}
\acro{ESB}{Enterprise Service Bus}
\end{acronym}
