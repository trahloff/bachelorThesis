\chapter{Introduction}

\begin{itemize}
        
    \item alle arbeitsschritte auch im inhaltsverzeichnis  
    
    \item kapitel exakt wie research method nennen 
    
    \item wichtige charakteristiken von serverless schon früh auf einem high-level  
    
    \item valider ansatz für das finden von requirements: charakteristik/requirement wird in literatur viel besprochen
        
    \item iot aus titel streichen und im fließtext nennen?
    
    \item noch länger einführen was es überhaupt ist
    
    \item einführung: ist es alles nur hype oder nicht ?
    
    \item es gibt zwei nonfunctional requirements: güte von viability/suitability. 
    am ende abwägen welche der lösung wie stark gewichtet
    
    \item nummerierung in schaubildern auch im text wiederfinden
    
    \item strukturen ganz wichtig. Figure \ref{fig:slessCompared} tatsächlich durchnummerieren. unterkapitel? itemize liste?
    
    \item generell mehr listenstrukturen 
    
\end{itemize}
%==========================================================%



\begin{minipage}{\textwidth}

    Within the following chapters, the presented research questions will be systematically addressed and answered. Hereby the thesis content is structured as follows:
    
    \begin{center}
        \begin{tabular}{ | m{11em} | m{23em}| } 
            \hline
             \textbf{Motivation} & 
             Why even bother? Why is a serverless approach worth a consideration?   \\
             \hline 
             
             \textbf{Research Method} &  
             The applied research model, the operationalization approach, research deliverables, design principles and result evaluation. \\
             \hline 
             
             \textbf{Background} & 
             research domain, context, placement \\
             \hline 
             
             \textbf{Prototype Design} & 
             functional and non-func requirements for comparison of models \\
             \hline 
             
             \textbf{Discussion} & 
             Viability, Suitability, existing limitations, as well as an assessment of the practical value of the presented method. \\
             \hline 
             
             \textbf{Conclusion} & 
             Final research results, recommendation for future research. \\
            \hline
        \end{tabular}
    \end{center}

\end{minipage}


