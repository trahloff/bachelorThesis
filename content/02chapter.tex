\chapter{Motivation}


%===================================================================================================%
\section{Introduction}
%===================================================================================================%


%===================================================================================================%
\section{Why Serverless?}
%===================================================================================================%


The \acf{IoT} is one of the major growth markets of the recent years and is expected to reach a global revenue of \$457B by the end of 2020, attaining a \acf{CAGR} of 28.5\% over four years.\autocite{Columbus20172017Forecasts} Looking at the most common usecases for IoT applications, it is clear that the prevailing real-life scenarios can be characterized by ubiquitous sensors numbering in the millions or even billions constantly monitoring physical objects, events and humans alike. Right now, these observations are often communicated to a cloud data center for various analyses that in turn trigger reactions to the observed events which aims to improve the efficiency and reliability of systems and generate valuable insights.\autocite{Yannuzzi2014KeyComputing} \\
This pattern results in a closed-loop \acf{OODA} cycle, where the the three major components are the information \textit{producers} (sensors), the \textit{cloud endpoint} and the \textit{consumer} or \textit{processor}.\autocite{Shukla2017BenchmarkingApplications} Especially this closed-loop characteristic of IoT applications is essential for their effective use and a low latency between observing events and processing them is therefore a fundamental requirement. To derive actionable insights that have a business impact, it is essential to process the data ingress in near real-time since information often have a time-to-live are only valid for a short amount of time, hence a rapidly scaling processor-concept that is performant enough to digest the data ingress is an imperative system component. \acf{ESP} approaches are an obvious solution to this challenge and reference IoT solutions from cloud providers \footnote{\url{https://aws.amazon.com/iot-core/features/}}\textsuperscript{,}\footnote{\url{https://microsoft.com/en-in/cloud-platform}} include some sort of event streaming.


%===================================================================================================%
\section{Scientific Body of Knowledge}
%===================================================================================================%


The purpose of this research is to develop a framework for architects that enables them to decide if a "serverless" architectural approach is fitted for their stream-processing use-case.
One major objective is to asses the current industry understanding of "serverless" and evaluate the viability and range of application of this architecture pattern.
Moreover, it is supposed to assist the process of highlevel system design by providing a reference and guidance by introducing fPaaS\footnote{"Function-Platform-as-a-Service". Defined by Gartner \autocite{Chandrasekaran2017EvolutionWhen}} capabilities and caveats. 



%===================================================================================================%
\section{Research Gap/Question}
%===================================================================================================%




%===================================================================================================%
\section{Content and Thesis Structure}
%===================================================================================================%

Within the following chapters, the presented research questions will be systematically addressed and answered. Hereby the thesis content is structured as follows:

\begin{center}
\begin{tabular}{ | m{11em} | m{23em}| } 
\hline
 \textbf{Research Method} & 
 The applied research model, the operationalization approach, research deliverables, design principles and result evaluation.  \\
 \hline 
 
 \textbf{Theoretical Background} & 
 A general introduction into the cloud-computing domain and "Serverless" architectures, as well as a systematic literature review on both topics. \\
 \hline 
 
 \textbf{Design} & 
 conduct the TTF/FVM \\
 \hline 
 
 \textbf{Validation} & 
 Use case driven validation of method applicability and method usability by open source experts. \\
 \hline 
 
 \textbf{Discussion} & 
 Final research results, existing limitations, recommendation for future research, as well as an assessment of the practical value of the presented method. \\
 \hline 
 
 \textbf{Bibliography} & 
 Cited scientific literature throughout the thesis. \\
 \hline 
 
 \textbf{Appendices} & 
 Deliverables, too large or unsuitable to be placed within the main body of the research document. \\
\hline
\end{tabular}
\end{center}


%===================================================================================================%
\section{Summary}
%===================================================================================================%