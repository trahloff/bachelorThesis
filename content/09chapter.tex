\chapter{Suitability Assessment}\label{chp:suitabilityAssessment}


%===================================================================================================%
\section{Introduction}
%===================================================================================================%

In chapter~\vref{chp:operationalization} the two system characteristics to evaluate \textit{Automated/Limitless Scaling} and \textit{Cost} were identified, specified, selected and validated. This chapter addresses the evaluation of the \textit{Suitability} domain and thevrefore focuses on the \textit{Automated/Limitless Scaling} property of the prototypes that are evaluated. Based on the system's requirements and thevrefore task and environment characteristics in chapter~\vref{chp:operationalization}, test cases and prototypes to run against the test cases have been designed in chapter~\vref{chp:prototyping}.

In the following, these suitability will be discussed.

%===================================================================================================%
\section{Non-Serverless}\label{chp:suitNSL}
%===================================================================================================%

The scalability of the containerized approach is restricted due to the constraint that instances have to be pre-provisioned and cannot be allocated dynamically. As outlined in chapter~\vref{chp:operationalization} and chapter~\vref{chp:prototyping}, a total requirement of 10 instances has been calculated. 

\begin{figure}[ht]
    \begin{tikzpicture}
        \begin{axis}[
                ,xlabel         = $\# Requests$
                ,ylabel         = {$Provisioned \, Compute \, Power$}
                ,xticklabels    = {,,}
                ,axis x line    = bottom
                ,axis y line    = left
                ,domain         = 0:5
                ]
            \addplot[no marks, color=blue]{10};
        \end{axis}
    \end{tikzpicture}\centering
    \caption {Provisioned Container Instances}
    \label{graph:provisionedContainerInstances}
\end{figure}

Therefore, an \textit{Automated/Limitless Scalability} is not achievable with the non-serverless approach outlined in this study. 

%===================================================================================================%
\section{Serverless}\label{chp:suitSL}
%===================================================================================================%

The serverless prototype scales completely automatic, event-driven, elastic and managed by the platform provider (AWS). Only the compute power needed to process the incoming messages is provisioned which can be seen in figure~\vref{graph:provisionedServerlessFunctions}. The 10,000 messages per second that are part of the evaluated test scenario were no problem at all. After consulting the AWS CloudWatch metrics, it is clear that the first few function invocations take a slightly increased amount of time to start up compared to later invoked functions. This is called a "cold-start"~\cfcite{Castro2017ServerlessService} and does not have a great impact on the average performance for this task because it only affects the very first function invocations. Overall, the average duration is approximately 60ms, which is consistent with the expected duration. 

\begin{figure}[ht]
    \begin{tikzpicture}
        \begin{axis}[
                ,xlabel         = $\# Requests$
                ,ylabel         = {$Provisioned \, Compute \, Power$}
                ,yticklabels    = {,,}
                ,xticklabels    = {,,}
                ,axis x line    = bottom
                ,axis y line    = left
                ,domain         = 0:5
                ]
            \addplot[no marks, color=blue]{x};
        \end{axis}
    \end{tikzpicture}\centering
    \caption {Provisioned Serverless Functions}
    \label{graph:provisionedServerlessFunctions}
\end{figure}

Not only does the serverless compute solution scales \textbf{up} with increasing data ingress, it scales \textbf{down} after the load was successfully processed, proofing that this solution is truly elastic and scales automatically with every incoming message. This can be achieved because for new messages, a new micro-container instance is spun up and destroyed after the message was processed. 


%===================================================================================================%
\section{Comparison}\label{chp:suitComp}
%===================================================================================================%

While the non-serverless approach is very inelastic and needs to be pre-provisioned to operate, the serverless approach scaled in an automated and limitless way. No configurations had to be done for the AWS Lambda prototype to achieve this scalability. 

%===================================================================================================%
\section{Summary}
%===================================================================================================%

An \textit{Automated/Limitless Scalability} is not achievable with the non-serverless approach outlined in this study. 